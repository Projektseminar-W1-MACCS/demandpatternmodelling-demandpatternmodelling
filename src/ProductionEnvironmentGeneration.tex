\documentclass[]{article}
\usepackage{lmodern}
\usepackage{amssymb,amsmath}
\usepackage{ifxetex,ifluatex}
\usepackage{fixltx2e} % provides \textsubscript
\ifnum 0\ifxetex 1\fi\ifluatex 1\fi=0 % if pdftex
  \usepackage[T1]{fontenc}
  \usepackage[utf8]{inputenc}
\else % if luatex or xelatex
  \ifxetex
    \usepackage{mathspec}
  \else
    \usepackage{fontspec}
  \fi
  \defaultfontfeatures{Ligatures=TeX,Scale=MatchLowercase}
\fi
% use upquote if available, for straight quotes in verbatim environments
\IfFileExists{upquote.sty}{\usepackage{upquote}}{}
% use microtype if available
\IfFileExists{microtype.sty}{%
\usepackage{microtype}
\UseMicrotypeSet[protrusion]{basicmath} % disable protrusion for tt fonts
}{}
\usepackage[margin=1in]{geometry}
\usepackage{hyperref}
\hypersetup{unicode=true,
            pdftitle={ProductionEnvironmentGeneration.R},
            pdfauthor={kaigm},
            pdfborder={0 0 0},
            breaklinks=true}
\urlstyle{same}  % don't use monospace font for urls
\usepackage{color}
\usepackage{fancyvrb}
\newcommand{\VerbBar}{|}
\newcommand{\VERB}{\Verb[commandchars=\\\{\}]}
\DefineVerbatimEnvironment{Highlighting}{Verbatim}{commandchars=\\\{\}}
% Add ',fontsize=\small' for more characters per line
\usepackage{framed}
\definecolor{shadecolor}{RGB}{248,248,248}
\newenvironment{Shaded}{\begin{snugshade}}{\end{snugshade}}
\newcommand{\AlertTok}[1]{\textcolor[rgb]{0.94,0.16,0.16}{#1}}
\newcommand{\AnnotationTok}[1]{\textcolor[rgb]{0.56,0.35,0.01}{\textbf{\textit{#1}}}}
\newcommand{\AttributeTok}[1]{\textcolor[rgb]{0.77,0.63,0.00}{#1}}
\newcommand{\BaseNTok}[1]{\textcolor[rgb]{0.00,0.00,0.81}{#1}}
\newcommand{\BuiltInTok}[1]{#1}
\newcommand{\CharTok}[1]{\textcolor[rgb]{0.31,0.60,0.02}{#1}}
\newcommand{\CommentTok}[1]{\textcolor[rgb]{0.56,0.35,0.01}{\textit{#1}}}
\newcommand{\CommentVarTok}[1]{\textcolor[rgb]{0.56,0.35,0.01}{\textbf{\textit{#1}}}}
\newcommand{\ConstantTok}[1]{\textcolor[rgb]{0.00,0.00,0.00}{#1}}
\newcommand{\ControlFlowTok}[1]{\textcolor[rgb]{0.13,0.29,0.53}{\textbf{#1}}}
\newcommand{\DataTypeTok}[1]{\textcolor[rgb]{0.13,0.29,0.53}{#1}}
\newcommand{\DecValTok}[1]{\textcolor[rgb]{0.00,0.00,0.81}{#1}}
\newcommand{\DocumentationTok}[1]{\textcolor[rgb]{0.56,0.35,0.01}{\textbf{\textit{#1}}}}
\newcommand{\ErrorTok}[1]{\textcolor[rgb]{0.64,0.00,0.00}{\textbf{#1}}}
\newcommand{\ExtensionTok}[1]{#1}
\newcommand{\FloatTok}[1]{\textcolor[rgb]{0.00,0.00,0.81}{#1}}
\newcommand{\FunctionTok}[1]{\textcolor[rgb]{0.00,0.00,0.00}{#1}}
\newcommand{\ImportTok}[1]{#1}
\newcommand{\InformationTok}[1]{\textcolor[rgb]{0.56,0.35,0.01}{\textbf{\textit{#1}}}}
\newcommand{\KeywordTok}[1]{\textcolor[rgb]{0.13,0.29,0.53}{\textbf{#1}}}
\newcommand{\NormalTok}[1]{#1}
\newcommand{\OperatorTok}[1]{\textcolor[rgb]{0.81,0.36,0.00}{\textbf{#1}}}
\newcommand{\OtherTok}[1]{\textcolor[rgb]{0.56,0.35,0.01}{#1}}
\newcommand{\PreprocessorTok}[1]{\textcolor[rgb]{0.56,0.35,0.01}{\textit{#1}}}
\newcommand{\RegionMarkerTok}[1]{#1}
\newcommand{\SpecialCharTok}[1]{\textcolor[rgb]{0.00,0.00,0.00}{#1}}
\newcommand{\SpecialStringTok}[1]{\textcolor[rgb]{0.31,0.60,0.02}{#1}}
\newcommand{\StringTok}[1]{\textcolor[rgb]{0.31,0.60,0.02}{#1}}
\newcommand{\VariableTok}[1]{\textcolor[rgb]{0.00,0.00,0.00}{#1}}
\newcommand{\VerbatimStringTok}[1]{\textcolor[rgb]{0.31,0.60,0.02}{#1}}
\newcommand{\WarningTok}[1]{\textcolor[rgb]{0.56,0.35,0.01}{\textbf{\textit{#1}}}}
\usepackage{graphicx,grffile}
\makeatletter
\def\maxwidth{\ifdim\Gin@nat@width>\linewidth\linewidth\else\Gin@nat@width\fi}
\def\maxheight{\ifdim\Gin@nat@height>\textheight\textheight\else\Gin@nat@height\fi}
\makeatother
% Scale images if necessary, so that they will not overflow the page
% margins by default, and it is still possible to overwrite the defaults
% using explicit options in \includegraphics[width, height, ...]{}
\setkeys{Gin}{width=\maxwidth,height=\maxheight,keepaspectratio}
\IfFileExists{parskip.sty}{%
\usepackage{parskip}
}{% else
\setlength{\parindent}{0pt}
\setlength{\parskip}{6pt plus 2pt minus 1pt}
}
\setlength{\emergencystretch}{3em}  % prevent overfull lines
\providecommand{\tightlist}{%
  \setlength{\itemsep}{0pt}\setlength{\parskip}{0pt}}
\setcounter{secnumdepth}{0}
% Redefines (sub)paragraphs to behave more like sections
\ifx\paragraph\undefined\else
\let\oldparagraph\paragraph
\renewcommand{\paragraph}[1]{\oldparagraph{#1}\mbox{}}
\fi
\ifx\subparagraph\undefined\else
\let\oldsubparagraph\subparagraph
\renewcommand{\subparagraph}[1]{\oldsubparagraph{#1}\mbox{}}
\fi

%%% Use protect on footnotes to avoid problems with footnotes in titles
\let\rmarkdownfootnote\footnote%
\def\footnote{\protect\rmarkdownfootnote}

%%% Change title format to be more compact
\usepackage{titling}

% Create subtitle command for use in maketitle
\providecommand{\subtitle}[1]{
  \posttitle{
    \begin{center}\large#1\end{center}
    }
}

\setlength{\droptitle}{-2em}

  \title{ProductionEnvironmentGeneration.R}
    \pretitle{\vspace{\droptitle}\centering\huge}
  \posttitle{\par}
    \author{kaigm}
    \preauthor{\centering\large\emph}
  \postauthor{\par}
      \predate{\centering\large\emph}
  \postdate{\par}
    \date{2019-08-30}


\begin{document}
\maketitle

\begin{Shaded}
\begin{Highlighting}[]
\CommentTok{#############################################################}
\CommentTok{# PRODUCTION ENVIRONMENT GENERATION V 0.01}
\CommentTok{# }
\CommentTok{#############################################################}

\NormalTok{gen_ProductionEnvironment <-}\StringTok{ }\ControlFlowTok{function}\NormalTok{(PRODUCTION_ENVIRONMENT) \{}

  \CommentTok{#  if vary_demand==0 }
 \CommentTok{#  DEMAND_BASE = lognrnd(1,VOL_VAR,NUMB_PRO,1);}
 \CommentTok{#  else}
 \CommentTok{#  rng('shuffle') }


\CommentTok{## ====================== STEP 1 REALIZED DEMAND GENERATION ========================= }

\NormalTok{units =}\StringTok{ }\DecValTok{10}\OperatorTok{^}\DecValTok{3}
\NormalTok{preDemand =}\StringTok{ }\KeywordTok{rlnorm}\NormalTok{(NUMB_PRO, }\DataTypeTok{meanlog =} \DecValTok{0}\NormalTok{, }\DataTypeTok{sdlog =} \FloatTok{0.1}\NormalTok{)}
\NormalTok{DEMAND =}\StringTok{ }\KeywordTok{ceiling}\NormalTok{(preDemand}\OperatorTok{/}\KeywordTok{sum}\NormalTok{(preDemand)}\OperatorTok{*}\NormalTok{units)}

\KeywordTok{barplot}\NormalTok{(}\KeywordTok{sort}\NormalTok{(DEMAND))}


 \CommentTok{# ## ====================== STEP 1  Determining the ACT_CONS_PA  =========================}
 \CommentTok{#    }
 \CommentTok{#  ProductionEnvironment.NUMB_RES = NUMB_RES; #Amount of processes}
 \CommentTok{#  ProductionEnvironment.NUMB_PRO = NUMB_PRO; #Amount of products}
 \CommentTok{#  }
 \CommentTok{#  }
 \CommentTok{# ## ====================== STEP 2 Determining the amount of cost categories (fix vs. variable costs) =========================}
 \CommentTok{#    }
 \CommentTok{#  if VOL_SHARE_RES == -1}
 \CommentTok{#  VOL_SHARE_RES_MIN = 0.3; }
 \CommentTok{#  VOL_SHARE_RES_MAX = 0.7;}
 \CommentTok{#  VOL_SHARE_RES = VOL_SHARE_RES_MIN + (VOL_SHARE_RES_MAX-VOL_SHARE_RES_MIN).*rand(1,1);}
 \CommentTok{#  else}
 \CommentTok{#  end }
 \CommentTok{#  }
 \CommentTok{#  ProductionEnvironment.UnitSize=floor(VOL_SHARE_RES*ProductionEnvironment.NUMB_RES);}
 \CommentTok{#  ProductionEnvironment.BatchSize=ProductionEnvironment.NUMB_RES-ProductionEnvironment.UnitSize;}
 \CommentTok{#  }
 \CommentTok{# ## ====================== STEP 2.b Determining a DMM (RES_CONS_PAT);  =========================}
 \CommentTok{#    }
 \CommentTok{#    %% Randomization and setting clear design points. }
 \CommentTok{#  if DENS == -1}
 \CommentTok{#  DENS_MIN = 0.4; }
 \CommentTok{#  DENS_MAX = 0.7;}
 \CommentTok{#  DENS_RUN = DENS_MIN + (DENS_MAX-DENS_MIN).*rand(1,1);}
 \CommentTok{#  else}
 \CommentTok{#    DENS_MIN = DENS-0.1;}
 \CommentTok{#  DENS_MAX = DENS;}
 \CommentTok{#  DENS_RUN = DENS_MIN + (DENS_MAX-DENS_MIN).*rand(1,1);}
 \CommentTok{#  end }
 \CommentTok{#  }
 \CommentTok{#  }
 \CommentTok{#  }
 \CommentTok{#  %DENS_RUN = 1 ; }
 \CommentTok{#  }
 \CommentTok{#  }
 \CommentTok{#  [RES_CONS_PAT,CHECK] = genRES_CONS_PAT(ProductionEnvironment,DENS_RUN,COR); % generate res_cons_pat}
 \CommentTok{#  %[RES_CONS_PAT,CHECK] = genRES_CONS_PAT2(ProductionEnvironment,DENS_RUN,COR); % generate res_cons_pat}
 \CommentTok{#  }
 \CommentTok{#  ## ====================== STEP 2.b Determining a DMM (RES_CONS_PAT) ===========================}
 \CommentTok{#    }
 \CommentTok{#    RC = genRC(ProductionEnvironment,VOL_SHARE_RES,RC_VAR,TC);}
 \CommentTok{#  [RES_CONS_PATp,CostSystem,CHECK] = genCOST_CONS_PAT(ProductionEnvironment,CHECK,RC,RES_CONS_PAT,DENS_RUN,COR);}
 \CommentTok{#  }
 \CommentTok{#  }
 \CommentTok{#  }
 \CommentTok{#  %% COMPUTING DESCRIPTIVE VALUES }
 \CommentTok{#  % Computing Resource cost percentage: How many percentage are in the}
 \CommentTok{#  RC_sort = sort(RC,'descend');}
 \CommentTok{#  RC_20p = sum(RC_sort(1:(NUMB_RES*0.2)));}
 \CommentTok{#  CHECK.RC_20p = RC_20p./TC*100;}
 \CommentTok{#  }
 \CommentTok{#  % Computing Output distribution percentage; How many percentage are in the % 20% values?}
 \CommentTok{#    DEMAND_sort = sort(TQ,'descend');}
 \CommentTok{#  DEMAND_20p = sum(DEMAND_sort(1:(NUMB_PRO*0.2))); }
 \CommentTok{#  CHECK.DEMAND_20p = DEMAND_20p./sum(TQ)*100;}
 \CommentTok{#  }
 \CommentTok{#  % Computing heterogeneity }
 \CommentTok{#  [CHECK] = measure_heterogeneity(RES_CONS_PATp,ProductionEnvironment,CHECK);}
 \CommentTok{#  }
 \CommentTok{#  % Averange range between lowest and highest consumption.}
 \CommentTok{#  CORAP_pre=max(RES_CONS_PATp)-min(RES_CONS_PATp);}
 \CommentTok{#  CHECK.CORAP=mean(CORAP_pre)*100;  }
 \CommentTok{#  }
\KeywordTok{return}\NormalTok{(PRODUCTION_ENVIRONMENT)}
\NormalTok{  \} }\CommentTok{# Function end}
\end{Highlighting}
\end{Shaded}


\end{document}
